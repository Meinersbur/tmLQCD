\section{Output files}

\subsection*{\ttfamily output.data}

The file {\ttfamily output.data} contains lines for each performed
trajectory. Each line has entries with the following meaning:
\begin{enumerate}
\item Plaquette value.
\item $\Delta H$
\item $\exp(-\Delta H)$
\item number of pseudo fermion fields times three integers. The first
  of the three is the number of CG (BiCGstab)  iterations used in the acceptance
  step, the second is the number of CG (BiCGstab) iterations used for
  the force computation and the third -- only different from $0$ if
  the BiCGstab is used as solver -- the BiCGstab iterations used for
  the second inversion in the force computation needed if BiCGstab is
  used.
\item Acceptance.
\item Time in second needed for this trajectory. In case of non MPI
  this is zero, because not measured.
\item Value of the rectangle part in the gauge action, if used.
\end{enumerate}
Every new run will append its numbers to an allready existing file.

\subsection*{\ttfamily output.para}
This file contains the parameters used in this run. Old files will be
overwritten. 

\subsection*{\ttfamily history\_hmc\_tm}
This file provides a mapping between the configuration number and its
plaquette and Poliakov loop values. Moreover the simulation parameters
are stored there and in case of a reread the time point can be found there. 

\subsection*{\ttfamily solver\_data}
Contains information about the iterations, the residuum, the $\mu$
values and the square norm of the source for each inversion.

\subsection*{\ttfamily return\_check.data}
Contains the reversibility violation measurements, if they are
performed. 

\subsection*{\ttfamily conf.save, rlxd\_state}
These two files are written after each trajectory giving the
possibility to continue the program from them. When the program
finishes it writes in addition the files {\ttfamily
  last\_configuration} and {\ttfamily last\_state} which might also be
used for a continuation.

%%% Local Variables: 
%%% mode: latex
%%% TeX-master: "main"
%%% End: 
