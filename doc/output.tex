\subsection{Output files}

\subsubsection*{\ttfamily output.data}

The file {\ttfamily output.data} contains lines for each performed
trajectory. Each line has entries with the following meaning:
\begin{enumerate}
\item Plaquette value.
\item $\Delta H$
\item $\exp(-\Delta H)$
\item number of pseudo fermion monomials times two integers. The first
  of the three is the number of CG or BiCGstab solveriterations used
  in the acceptance and heatbath steps, the second is the number of CG
  (BiCGstab) iterations used for the force computation.
\item Acceptance.
\item Time in seconds needed for this trajectory. In case of non MPI
  this is zero, because not measured.
\item Value of the rectangle part in the gauge action, if used.
\end{enumerate}
Every new run will append its numbers to an allready existing file.

\subsubsection*{\ttfamily output.para}
This file contains the parameters used in this run. Old files will be
overwritten. 

\subsubsection*{\ttfamily history\_hmc\_tm}
This file provides a mapping between the configuration number and its
plaquette and Poliakov loop values. Moreover the simulation parameters
are stored there and in case of a reread the time point can be found there. 

\subsubsection*{\ttfamily return\_check.data}
Contains the reversibility violation measurements, if they are
performed. 

\subsubsection*{\ttfamily conf.save}
This file is written after each trajectory, if no regular
configuration is saved. It contains the most recent gauge
configuration and the status of the random number generator for a
restart of the programme. 

\subsubsection*{\ttfamily onlinemeas.N}
Contains the online measurement for trajectory {\ttfamily N} if this
feature is switched on.

%%% Local Variables: 
%%% mode: latex
%%% TeX-master: "main"
%%% End: 
