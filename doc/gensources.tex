\section{Programme {\ttfamily gen\_sources}}

The programme {\ttfamily gen\_sources} provides an interface to
generate stochastic sources for several different situations. It is
able to generate those for the nucleon case (which should not be used,
because point sources are optimal), for mesons in general and for the
special case of the pion only.

The programme offers command line options as follows:
\begin{itemize}
\item {\ttfamily -h|?} a help.
\item {\ttfamily -L} the spatical lattice size
\item {\ttfamily -T} the temporal lattice size
\item {\ttfamily -o} the base filename of the sources (default is
  {\ttfamily source})
\item {\ttfamily -n} the configuration number (default is $0$)
\item {\ttfamily -s} the sample number (default is $0$)
\item {\ttfamily -t} the value of the start timeslice (default $0$)
\item {\ttfamily -S} the spatial spacing/dilution (default $1$)
\item {\ttfamily -P} the temporal spacing/dilution (default $T$)
\item {\ttfamily -N} produce nucleon sources (default meson sources)
\item {\ttfamily -p} plain output filename (see below)
\item {\ttfamily -O} the special pion only case
\item {\ttfamily -E} extended sources for pion three point
  functions. Together with {\ttfamily -O}
\item {\ttfamily -d} write source in double precision (default single)
\item {\ttfamily -a} write all sources in one file rather than $12$
  (pion only is one file anyhow)
\end{itemize}
The output filename is generated like {\ttfamily
  base.sampleno.gaugeno.tsno.00 -11}, unless {\ttfamily -p} is chosen,
which would correspond to {\ttfamily base.00-11}.

The special pion only case corresponds to a single timeslice source
without any dilution in spin or colour or space.

\endinput
%%% Local Variables: 
%%% mode: latex
%%% TeX-master: "main"
%%% End: 
