\section{Integration schemes}
\label{sec:integrators}

Most of the details for the integration schemes can be found in
{\ttfamily hep-lat/0506011}. Therefore we give here only details for
the symplectic integration schemes {\ttfamily 2MN} and {\ttfamily
  2MNposition}. 

The second order minimal norm (2MN) integration scheme is a
generalisation of the Sexton-Weingarten scheme. While the latter is
build by the baisc integration step:
\begin{equation}
  \label{int:0}
  T_{\mathrm{SW}_0}\ =\ T_{\mathrm{S}_0}(\dtau_0/6)\
  T_\mathrm{U}(\dtau_0/2)\ T_{\mathrm{S}_0}(2\dtau_0/3)\
  T_\mathrm{U}(\dtau_0/2)\ T_{\mathrm{S}_0}(\dtau_0/6)\, ,
\end{equation}
the 2MN scheme is build onto
\begin{equation}
  \label{int:1}
  T_{\mathrm{2MN}_0}\ =\ T_{\mathrm{S}_0}(\lambda_0\dtau_0)\
  T_\mathrm{U}(\dtau_0/2)\ T_{\mathrm{S}_0}((1-2\lambda_0)\dtau_0)\
  T_\mathrm{U}(\dtau_0/2)\ T_{\mathrm{S}_0}(\lambda_0\dtau_0)\, .
\end{equation}
$\lambda_0$ is a dimensionless parameter and the 2MN coincides with
the Sexton-Weingarten scheme in case $\lambda_0=1/6$. The optimal
value for $\lambda_0$ was given in Ref.~\cite{Takaishi:2005tz} to be
around $0.19$. But its value is likely to depend on the mass values
and the time scale under consideration.

We can now introduce a parameter $\lambda_i$ for each timescale
$\dtau_i$ and tune them seperatly.

%%% Local Variables: 
%%% mode: latex
%%% TeX-master: "main"
%%% End: 
