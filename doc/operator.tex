\section{Dirac operator}

It is well known that $(1 \pm \gamma_\nu)\psi$ has only two
independent components from which the other two can be
recomputed. This can be used on the one hand to reduce the
computational effort by multiplying only the two independent
components with the gauge field and reconstruct the other two from
them. On the other hand it can be used to reduce communication
overhead by communicating only a half spinor. 

The first trick is always used in the code, whereas the second one is
implemented in a second version of the Dirac operator. It can be used
by configuring with {\ttfamily --enable-newdiracop} ({\ttfamily
  --enable-gaugecopy} is the automatically switched on). 

The idea of this new version is the following:
hopping-matrix times vector $\chi=H\psi$ in three steps
\begin{enumerate}
\item<1-> project to a half Spinor $\phi$:
  \[
  \phi = H_{4\to 2}\psi
  \]
  Keep only the two independent elements of $(1\pm\gamma_i)\psi$.
\item<2-> exchange halfspinor $\phi$
\item<3-> expand to full Spinor:
  \[
  \chi = H_{2\to 4}\phi
  \]
\end{enumerate}
The multiplication with the gauge fields is done at the moment
symmetrically in the sense that in step 1 the plus direction is
multiplied and in step 3 the minus direction. 

Depending on the platform this new version gives a quite reasonable
speedup, in particular on the SGI Altix and the BG/L.

\subsection{Sloppy precision}

An additional way to speed up the Dirac operator is to store the field
$\phi$ only in single precision. This reduces the memory traffic and
the communication traffic significantly. This trick can be used for
instance in the MD part of the HMC without loosing the exactness of
the algorithm. However, reversibility violations might be an issue.

Using this trick in the MD part can be triggered using the input
parameter {\ttfamily UseSloppyPrecision=yes}. It works quite well and
gives a speedup of about 30\%.

%%% Local Variables: 
%%% mode: latex
%%% TeX-master: "main"
%%% End: 
