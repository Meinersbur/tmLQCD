\section*{Even-Odd preconditioning}

\begin{equation}
  \label{eq:eo0}
  M_{xy} =\delta_{xy}(1+i\mu\gamma_5\tau^3) - \kappa\sum_\mu \delta_{x,y+\hat\mu}(1+\gamma_\mu) 
\end{equation}
with $\mu \to \mu/(2\kappa)$.
\begin{equation}
  \label{eq:eo1}
  Q\equiv \gamma_5 M = \begin{pmatrix}
      Q_+ & \\\
         & Q_- \\
      \end{pmatrix}
\end{equation}
where
\begin{equation}
  \label{eq:eo2}
  \begin{split}
    Q_\pm &= \gamma_5\begin{pmatrix}
      1\pm i\mu\gamma_5 & M_{eo} \\
      M_{oe}    & 1\pm i\mu\gamma_5 \\
    \end{pmatrix} =
    \gamma_5\begin{pmatrix}
      M_{ee}^\pm & M_{eo} \\
      M_{oe}    & M_{oo}^\pm \\
    \end{pmatrix} \\
    & =
    \begin{pmatrix}
      \gamma_5M_{ee}^\pm & 0 \\
      \gamma_5M_{oe}  & 1 \\
    \end{pmatrix}
    \begin{pmatrix}
      1       & (M_{ee}^\pm)^{-1}M_{eo}\\
      0       & \gamma_5(M_{oo}^\pm-M_{oe}(M_{ee}^\pm)^{-1}M_{eo})\\
    \end{pmatrix}
\end{split}
\end{equation}
$(M_{ee}^\pm)^{-1}$ can be computed to 
\[
(1\pm i\mu\gamma_5)^{-1} = \frac{1\mp i\mu\gamma_5}{1+\mu^2}.
\]
With this we get
\begin{equation}
  \label{eq:eo3}
  \det(Q)=\det(Q_+)\det(Q_-) = \det(Q_+ Q_-).
\end{equation}
and appart from an irrelevant factor
\begin{equation}
  \label{eq:eo4}
  \begin{split}
    \det(Q_\pm) &\propto \det(\hat Q_\pm) \\
    \hat Q_\pm &= \gamma_5(M_{oo}^\pm - M_{oe}(M_{ee}^\pm )^{-1}M_{eo}). 
  \end{split}
\end{equation}
$\hat Q_\pm$ is only defined on the odd sides. Now we write the
determinant with pseudo fermion fields:
\[
\det(Q) = \det(\hat Q_+ \hat Q_-) \propto \int D[\phi]\exp(-\phi^\dagger (\hat Q_+ \hat Q_-)^{-1} \phi) = \int D[\phi]\exp(-S_b)
\]
For the HMC we need the Variation of $S_b$ (using $\delta (A^{-1})=A^{-1}\delta A A^{-1}$):
\begin{equation}
  \label{eq:eo5}
  \begin{split}
    \delta S_b &= -[\phi_o^\dagger (\hat Q_+ \hat Q_-)^{-1}\delta \hat Q_+ \hat Q_+^{-1}\phi_o +
    \phi_o^\dagger\hat Q_-^{-1}\delta \hat Q_- (\hat Q_+ \hat Q_-)^{-1} \phi_o ] \\
     &= -[X_o^\dagger \delta \hat Q_+ Y_o + Y_o^\dagger \delta\hat Q_- X_o]
  \end{split}
\end{equation}
with $X_o$ and $Y_o$ defined on the odd sides as 
\begin{equation}
  \label{eq:eo6}
  X_o = (\hat Q_+ \hat Q_-)^{-1} \phi_o,\quad Y_o = \hat Q_+^{-1}\phi_o=\hat
  Q_-X_o\ ,
\end{equation}
using
\[
\hat Q_\pm^\dagger = \hat Q_\mp\, .
\]
The variation of $\hat Q_\pm$ reads
\begin{equation}
  \label{eq:eo7}
  \delta \hat Q_\pm = \gamma_5\left(-\delta M_{oe}(M_{ee}^\pm )^{-1}M_{eo} -
    M_{oe}(M_{ee}^\pm )^{-1}\delta M_{eo}\right),
\end{equation}
and we find
\begin{equation}
  \label{eq:eo8}
  \delta S_b = -(X^\dagger\delta Q_+ Y + Y^\dagger\delta Q_- X) = -(X^\dagger\delta Q_+ Y +(X^\dagger\delta Q_+ Y)^\dagger)
\end{equation}
where $X,Y$ is now defined over the full lattice as
\begin{equation}
  \label{eq:eo9}
  X = 
  \begin{pmatrix}
    -(M_{ee}^-)^{-1}M_{eo}X_o \\ X_o\\
  \end{pmatrix},\quad
  Y = 
  \begin{pmatrix}
    -(M_{ee}^+)^{-1}M_{eo}Y_o \\ Y_o\\
  \end{pmatrix}.
\end{equation}
Additionally $\delta Q_+ = \delta Q_-$ holds and we used
\[
M_{eo}^\dagger = \gamma_5 M_{oe}\gamma_5\;\textrm{and}\; M_{oe}^\dagger = \gamma_5 M_{eo}\gamma_5.
\]
Since the bosonic part is quadratic in the $\phi_o$ fields, the $\phi_o$ are
generated at the beginnig of each molecular dynamics trajectory with
\begin{equation}
  \label{eq:eo10}
  \phi_o = \hat Q_+ R,
\end{equation}
where $R$ is a random spinor field taken from a Gaussian distribution
with norm one.

If one uses $\det(\hat Q_{-}\hat Q_{+})$ instead of $\det(\hat
Q_{+}\hat Q_{-})$ on gets
\begin{equation}
  \label{eq:eo11}
  \begin{split}
    \delta S_b & = -[\phi_o^\dagger(\hat Q_{-} \hat Q_{+})^{-1}\delta\hat Q_{-}\hat
    Q_{-}^{-1} \phi_o + \phi_o^\dagger \hat Q_{+}^{-1}\delta\hat Q_{+}(\hat
    Q_{-}\hat Q_{-})^{-1}\phi_o]\\
    & = -[X_o^\dagger\delta\hat Q_{-}Y_o + Y_O^\dagger\delta\hat Q_{+}X_o]
  \end{split}
\end{equation}
with
\begin{equation}
  \label{eq:eo12}
  X_o = (\hat Q_{-} \hat Q_{+})^{-1}\phi_o, \quad Y_o = \hat Q_{-}^{-1}\phi_o=\hat
  Q_{+} X_o.
\end{equation}
 We can reconstruct the spinors over the full lattice in the following
way:
\begin{equation}
  \label{eq:eo13}
  X = 
  \begin{pmatrix}
    -(M_{ee}^+)^{-1}M_{eo}X_o \\ X_o\\
  \end{pmatrix},\quad
  Y = 
  \begin{pmatrix}
    -(M_{ee}^-)^{-1}M_{eo}Y_o \\ Y_o\\
  \end{pmatrix},
\end{equation}
and the fields $\phi_o$ have to be constructed using $R$ via
\begin{equation}
  \label{eq:14}
  \phi_o = \hat Q_{-} R.
\end{equation}
Additionally 
\[
Y_o = \gamma_5(M_{ee}^{+}X_o + M_{oe} X_e)
\]
can be used to construct $Y_o$, if $X_e$ is allready computed. This
means one can compute $X_o = (\hat Q_{-} \hat Q_{+})^{-1}\phi_o$, then
$X_e = -(M_{ee}^+)^{-1}M_{eo}X_o$ followed by $Y_o = \gamma_5(M_{ee}^{+}X_o
+ M_{oe} X_e)$ and $Y_e = -(M_{ee}^-)^{-1}M_{eo}Y_o$, if one uses the
CG as inverter anyhow.

%%% Local Variables: 
%%% mode: latex
%%% TeX-master: "main"
%%% End: 
